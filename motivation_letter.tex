\documentclass[11pt, a4paper, oneside]{letter}
\usepackage[utf8]{inputenc}
\usepackage[T1]{fontenc}
\usepackage[russian]{babel}
\usepackage{geometry}
\usepackage{setspace}Ы
\usepackage{hyperref}

% Настройка полей
\geometry{top=2cm, bottom=2cm, left=3cm, right=3cm}

% Интервалы и отступы
\onehalfspacing % полуторный интервал
\setlength{\parindent}{1.25cm}  % красная строка
\setlength{\parskip}{0pt}       % убираем отступы между абзацами

\begin{document}

Уважаемые коллеги,

Меня зовут Андреева Анастасия, я обучаюсь на программе «Анализ данных в девелопменте» на факультете компьютерных наук. Я хотела бы выразить заинтересованность в возможности присоединиться к Международной лаборатории стохастического анализа и его приложений.

Мои научные интересы лежат в области: статистического анализа сложных структур при редких событиях (разреженные данные), асимптотического анализа и статистических алгоритмов для смесей распределений, прикладной эконометрики и финансовой математики (страхование, опционы, риски).

Несмотря на то, что мой исследовательский опыт пока ограничен, я выполнила небольшую самостоятельную работу в R для демонстрации своей мотивации (\href{https://github.com/entoshik/R/blob/main/danish_fire_claims_pareto_analysis.pdf}{ссылка на работу}). В ней я проанализировала данные о страховых выплатах по пожарам в Дании (датасет \texttt{danish}). Работа была посвящена выявлению закономерностей распределений с тяжелыми хвостами, характерных для крупных убытков.

Особое внимание я хотела бы уделить своим личным качествам, которые, на мой взгляд, важны для исследовательской деятельности. Я умею работать в команде, так как в диалоге часто рождаются наиболее интересные решения, а совместная работа расширяет кругозор. Я не боюсь ошибок, понимаю, что они являются неотъемлемой частью научного процесса, и стараюсь использовать их как возможность для обучения и совершенствования.

Я люблю и умею учиться. Мой высокий средний балл (GPA 9,2) является подтверждением того, что я способна к упорной работе, самообучению и освоению новых методов. Я стремлюсь расширять круг знаний, в том числе за пределами учебной программы, и хочу перенести эту настойчивость в исследовательскую практику.

Я была бы рада получить возможность учиться у специалистов вашей лаборатории, принять участие в текущих проектах и внести свой вклад в развитие методов стохастического анализа и их приложений к эконометрике, страхованию и финансовым рынкам.

\vspace{1cm}

\begin{flushright}
С уважением, \\
Андреева Анастасия
\end{flushright}

\end{document}
